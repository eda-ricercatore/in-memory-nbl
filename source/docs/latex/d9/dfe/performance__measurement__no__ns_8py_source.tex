\hypertarget{performance__measurement__no__ns_8py_source}{}\section{performance\+\_\+measurement\+\_\+no\+\_\+ns.\+py}
\label{performance__measurement__no__ns_8py_source}\index{utilities/timing\+\_\+measurements/performance\+\_\+measurement\+\_\+no\+\_\+ns.\+py@{utilities/timing\+\_\+measurements/performance\+\_\+measurement\+\_\+no\+\_\+ns.\+py}}

\begin{DoxyCode}
\hypertarget{performance__measurement__no__ns_8py_source_l00001}{}\hyperlink{namespaceutilities_1_1timing__measurements_1_1performance__measurement__no__ns}{00001} \textcolor{comment}{#!/usr/local/bin/python3}
00002 \textcolor{comment}{###!/Users/zhiyang/anaconda3/bin/python3}
00003 
00004 \textcolor{stringliteral}{"""}
00005 \textcolor{stringliteral}{    This Python script is written by Zhiyang Ong to measure the}
00006 \textcolor{stringliteral}{        execution time of functions and programs in Python, and}
00007 \textcolor{stringliteral}{        other processes and programs called from Python scripts}
00008 \textcolor{stringliteral}{        or programs.}
00009 \textcolor{stringliteral}{}
00010 \textcolor{stringliteral}{}
00011 \textcolor{stringliteral}{    Synopsis:}
00012 \textcolor{stringliteral}{    Measure the execution time of functions and programs in Python.}
00013 \textcolor{stringliteral}{}
00014 \textcolor{stringliteral}{    This script can be executed as follows:}
00015 \textcolor{stringliteral}{    ./performance\_measurement.py}
00016 \textcolor{stringliteral}{}
00017 \textcolor{stringliteral}{}
00018 \textcolor{stringliteral}{}
00019 \textcolor{stringliteral}{    Revision History:}
00020 \textcolor{stringliteral}{    September 6, 2019           Version 0.1 Script.}
00021 \textcolor{stringliteral}{"""}
00022 
\hypertarget{performance__measurement__no__ns_8py_source_l00023}{}\hyperlink{namespaceutilities_1_1timing__measurements_1_1performance__measurement__no__ns_ad21212ef2a99b878f1d71f04b282ee93}{00023} \_\_author\_\_ = \textcolor{stringliteral}{'Zhiyang Ong'}
\hypertarget{performance__measurement__no__ns_8py_source_l00024}{}\hyperlink{namespaceutilities_1_1timing__measurements_1_1performance__measurement__no__ns_aad2678d45f225046185eb85ebe3cdc6a}{00024} \_\_version\_\_ = \textcolor{stringliteral}{'1.0'}
\hypertarget{performance__measurement__no__ns_8py_source_l00025}{}\hyperlink{namespaceutilities_1_1timing__measurements_1_1performance__measurement__no__ns_a7b90e234766add970bbe1b7f69e543b9}{00025} \_\_date\_\_ = \textcolor{stringliteral}{'September 6, 2019'}
00026 
00027 \textcolor{comment}{#   The MIT License (MIT)}
00028 
00029 \textcolor{comment}{#   Copyright (c) <2019> <Zhiyang Ong>}
00030 
00031 \textcolor{comment}{#   Permission is hereby granted, free of charge, to any person obtaining a copy of this software and
       associated documentation files (the "Software"), to deal in the Software without restriction, including without
       limitation the rights to use, copy, modify, merge, publish, distribute, sublicense, and/or sell copies of the
       Software, and to permit persons to whom the Software is furnished to do so, subject to the following
       conditions:}
00032 
00033 \textcolor{comment}{#   The above copyright notice and this permission notice shall be included in all copies or substantial
       portions of the Software.}
00034 
00035 \textcolor{comment}{#   THE SOFTWARE IS PROVIDED "AS IS", WITHOUT WARRANTY OF ANY KIND, EXPRESS OR IMPLIED, INCLUDING BUT NOT
       LIMITED TO THE WARRANTIES OF MERCHANTABILITY, FITNESS FOR A PARTICULAR PURPOSE AND NONINFRINGEMENT. IN NO
       EVENT SHALL THE AUTHORS OR COPYRIGHT HOLDERS BE LIABLE FOR ANY CLAIM, DAMAGES OR OTHER LIABILITY, WHETHER IN AN
       ACTION OF CONTRACT, TORT OR OTHERWISE, ARISING FROM, OUT OF OR IN CONNECTION WITH THE SOFTWARE OR THE USE
       OR OTHER DEALINGS IN THE SOFTWARE.}
00036 
00037 \textcolor{comment}{#   Email address: echo "cukj -wb- 23wU4X5M589 TROJANS cqkH wiuz2y 0f Mw Stanford" | awk '\{
       sub("23wU4X5M589","F.d\_c\_b. ") sub("Stanford","d0mA1n"); print $5, $2, $8; for (i=1; i<=1; i++) print "6\(\backslash\)b"; print $9, $7,
       $6 \}' | sed y/kqcbuHwM62z/gnotrzadqmC/ | tr 'q' ' ' | tr -d [:cntrl:] | tr -d 'ir' | tr y "\(\backslash\)n"   Che cosa
       significa?}
00038 
00039 
00040 \textcolor{comment}{###############################################################}
00041 \textcolor{stringliteral}{"""}
00042 \textcolor{stringliteral}{    Import modules from The Python Standard Library.}
00043 \textcolor{stringliteral}{    sys         Get access to any command-line arguments.}
00044 \textcolor{stringliteral}{    os          Use any operating system dependent functionality.}
00045 \textcolor{stringliteral}{    os.path     For pathname manipulations.}
00046 \textcolor{stringliteral}{}
00047 \textcolor{stringliteral}{    subprocess -> call}
00048 \textcolor{stringliteral}{                To make system calls.}
00049 \textcolor{stringliteral}{    time        To measure elapsed time.}
00050 \textcolor{stringliteral}{    warnings    Raise warnings.}
00051 \textcolor{stringliteral}{    re          Use regular expressions.}
00052 \textcolor{stringliteral}{}
00053 \textcolor{stringliteral}{    pathlib->Path}
00054 \textcolor{stringliteral}{                For mapping a string to a path.}
00055 \textcolor{stringliteral}{    datetime    To obtain information about the current date and time.}
00056 \textcolor{stringliteral}{    time        To obtain information about the current time.}
00057 \textcolor{stringliteral}{                time\_ns version provides information about the}
00058 \textcolor{stringliteral}{                    current time with nanosecond accuracy.}
00059 \textcolor{stringliteral}{    warnings    Provide warnings to users without terminating the}
00060 \textcolor{stringliteral}{                    program abruptly.}
00061 \textcolor{stringliteral}{    process\_time (& process\_time\_ns)}
00062 \textcolor{stringliteral}{                To determine the time stamp using the process}
00063 \textcolor{stringliteral}{                    time method, which is platform independent in}
00064 \textcolor{stringliteral}{                    Python 3.x, and its alternative providing}
00065 \textcolor{stringliteral}{                    nanosecond accuracy.}
00066 \textcolor{stringliteral}{    perf\_counter (& perf\_counter\_ns)}
00067 \textcolor{stringliteral}{                To determine the time stamp using the process}
00068 \textcolor{stringliteral}{                    time method, which is platform independent in}
00069 \textcolor{stringliteral}{                    Python 3.x, and its alternative providing}
00070 \textcolor{stringliteral}{                    nanosecond accuracy.}
00071 \textcolor{stringliteral}{    monotonic (& pm\_monotonic\_ns)}
00072 \textcolor{stringliteral}{                To monotonically obtain time stamps, for performance}
00073 \textcolor{stringliteral}{                    measurement, and its alternative providing}
00074 \textcolor{stringliteral}{                    nanosecond accuracy.}
00075 \textcolor{stringliteral}{"""}
00076 
00077 \textcolor{keyword}{import} sys
00078 \textcolor{keyword}{import} os
00079 \textcolor{keyword}{import} os.path
00080 \textcolor{comment}{#from pathlib import Path}
00081 \textcolor{keyword}{from} subprocess \textcolor{keyword}{import} call
00082 \textcolor{keyword}{import} time
00083 \textcolor{keyword}{import} warnings
00084 \textcolor{keyword}{import} re
00085 \textcolor{keyword}{import} datetime
00086 \textcolor{comment}{# ImportError: cannot import name 'perf\_counter\_ns'}
00087 \textcolor{keyword}{from} time \textcolor{keyword}{import} perf\_counter \textcolor{keyword}{as} pc\_timestamp
00088 \textcolor{comment}{#from time import perf\_counter\_ns as pc\_timestamp\_ns}
00089 \textcolor{keyword}{from} time \textcolor{keyword}{import} process\_time \textcolor{keyword}{as} pt\_timestamp
00090 \textcolor{comment}{#from time import process\_time\_ns as pt\_timestamp\_ns}
00091 \textcolor{comment}{#from time import time\_ns as t\_ns}
00092 \textcolor{keyword}{from} time \textcolor{keyword}{import} monotonic \textcolor{keyword}{as} pm\_monotonic
00093 \textcolor{comment}{#from time import monotonic\_ns as pm\_monotonic\_ns}
00094 
00095 \textcolor{comment}{###############################################################}
00096 \textcolor{comment}{#   Import Custom Python Packages and Modules}
00097 \textcolor{stringliteral}{"""}
00098 \textcolor{stringliteral}{    Module to calculate the factorial of a number.}
00099 \textcolor{stringliteral}{"""}
00100 \textcolor{comment}{#from get\_factorial import calculate\_factorial}
00101 \textcolor{keyword}{from} \hyperlink{namespaceutilities_1_1timing__measurements_1_1get__factorial}{utilities.timing\_measurements.get\_factorial} \textcolor{keyword}{import} 
      calculate\_factorial
00102 
00103 \textcolor{comment}{###############################################################}
00104 \textcolor{stringliteral}{"""}
00105 \textcolor{stringliteral}{    Module with methods that measure the execution time of functions}
00106 \textcolor{stringliteral}{        and programs in Python.}
00107 \textcolor{stringliteral}{}
00108 \textcolor{stringliteral}{    Support is not provided for storing multiple initial timestamps,}
00109 \textcolor{stringliteral}{        so that we can measure elapsed times from different initial}
00110 \textcolor{stringliteral}{        timestamps.}
00111 \textcolor{stringliteral}{        User have to call the functions (such as monotonic\_ns())}
00112 \textcolor{stringliteral}{            specifically, so that this Python module can be kept}
00113 \textcolor{stringliteral}{            simple and short.}
00114 \textcolor{stringliteral}{"""}
\hypertarget{performance__measurement__no__ns_8py_source_l00115}{}\hyperlink{classutilities_1_1timing__measurements_1_1performance__measurement__no__ns_1_1execution__time__measurement}{00115} \textcolor{keyword}{class }\hyperlink{classutilities_1_1timing__measurements_1_1performance__measurement__no__ns_1_1execution__time__measurement}{execution\_time\_measurement}:
00116     \textcolor{comment}{# Invalid timestamp.}
\hypertarget{performance__measurement__no__ns_8py_source_l00117}{}\hyperlink{classutilities_1_1timing__measurements_1_1performance__measurement__no__ns_1_1execution__time__measurement_ac0b6b9477824b6d0b0e1348864d55566}{00117}     invalid\_timestamp = -123456789012345678901234567890
00118     \textcolor{comment}{# Initial timestamp.}
\hypertarget{performance__measurement__no__ns_8py_source_l00119}{}\hyperlink{classutilities_1_1timing__measurements_1_1performance__measurement__no__ns_1_1execution__time__measurement_a75ac358ee6e04eba517fd6dd429c99a1}{00119}     initial\_timestamp = invalid\_timestamp
00120     \textcolor{comment}{# Types of performance measurement technique available.}
00121     \textcolor{comment}{#types\_of\_performance\_measurement\_technique =
       ("perf\_counter","perf\_counter\_ns","process\_time","process\_time\_ns","time","time\_ns","monotonic","monotonic\_ns")}
\hypertarget{performance__measurement__no__ns_8py_source_l00122}{}\hyperlink{classutilities_1_1timing__measurements_1_1performance__measurement__no__ns_1_1execution__time__measurement_aa73046ac445a5731739e52ae5034c4a1}{00122}     types\_of\_performance\_measurement\_technique = (\textcolor{stringliteral}{"perf\_counter"},\textcolor{stringliteral}{"process\_time"},\textcolor{stringliteral}{"time"},\textcolor{stringliteral}{"monotonic"})
00123     \textcolor{comment}{# ============================================================}
00124     \textcolor{comment}{##  Method to set the initial timestamp.}
00125     \textcolor{comment}{#}
00126     \textcolor{comment}{#   Use techniques for measuring performance (i.e., user}
00127     \textcolor{comment}{#       execution time) and timestamps.}
00128     \textcolor{comment}{#}
00129     \textcolor{comment}{#}
00130     \textcolor{comment}{#}
00131     \textcolor{comment}{#   @param type\_timestamp - Indicates if either of the following}
00132     \textcolor{comment}{#               methods of performance measurement is preferred.}
00133     \textcolor{comment}{#               * perf\_counter, perf\_counter(): pc\_timestamp()}
00134     \textcolor{comment}{#               * process\_time, process\_time(): pt\_timestamp()}
00135     \textcolor{comment}{#               * time, time.time(): time()}
00136     \textcolor{comment}{#               * monotonic, monotonic(): pm\_monotonic()}
00137     \textcolor{comment}{#   @return - Nothing.}
00138     \textcolor{comment}{#   O(1) method.}
00139     @staticmethod
\hypertarget{performance__measurement__no__ns_8py_source_l00140}{}\hyperlink{classutilities_1_1timing__measurements_1_1performance__measurement__no__ns_1_1execution__time__measurement_ac1a396608592993d871613a73d20b088}{00140}     \textcolor{keyword}{def }\hyperlink{classutilities_1_1timing__measurements_1_1performance__measurement__no__ns_1_1execution__time__measurement_ac1a396608592993d871613a73d20b088}{set\_initial\_timestamp}(type\_timestamp="monotonic"):
00141         \textcolor{stringliteral}{"""}
00142 \textcolor{stringliteral}{            Is the option for one of the following methods to measure}
00143 \textcolor{stringliteral}{                time?}
00144 \textcolor{stringliteral}{                * perf\_counter, perf\_counter(): pc\_timestamp()}
00145 \textcolor{stringliteral}{                * process\_time, process\_time(): pt\_timestamp()}
00146 \textcolor{stringliteral}{                * time, time.time(): time.time()}
00147 \textcolor{stringliteral}{                * monotonic, monotonic(): pm\_monotonic()}
00148 \textcolor{stringliteral}{        """}
00149         \textcolor{keywordflow}{if} (\textcolor{stringliteral}{"perf\_counter"} == type\_timestamp):
00150             \textcolor{comment}{# Yes. Use perf\_counter() to measure performance/time.}
00151             execution\_time\_measurement.initial\_timestamp = pc\_timestamp()
00152         \textcolor{keywordflow}{elif} (\textcolor{stringliteral}{"process\_time"} == type\_timestamp):
00153             \textcolor{comment}{# Yes. Use process\_time() to measure performance/time.}
00154             execution\_time\_measurement.initial\_timestamp = pt\_timestamp()
00155         \textcolor{keywordflow}{elif} (\textcolor{stringliteral}{"time"} == type\_timestamp):
00156             \textcolor{comment}{# Yes. Use time() to measure performance/time.}
00157             execution\_time\_measurement.initial\_timestamp = time.time()
00158         \textcolor{keywordflow}{else}:
00159             \textcolor{comment}{# The default option is: "monotonic()"}
00160             execution\_time\_measurement.initial\_timestamp = pm\_monotonic()
00161     \textcolor{comment}{# ============================================================}
00162     \textcolor{comment}{##  Method to get the initial timestamp.}
00163     \textcolor{comment}{#   @return the initial timestamp.}
00164     \textcolor{comment}{#   O(1) method.}
00165     @staticmethod
\hypertarget{performance__measurement__no__ns_8py_source_l00166}{}\hyperlink{classutilities_1_1timing__measurements_1_1performance__measurement__no__ns_1_1execution__time__measurement_af12752f53bfd297b66b504170b9e3655}{00166}     \textcolor{keyword}{def }\hyperlink{classutilities_1_1timing__measurements_1_1performance__measurement__no__ns_1_1execution__time__measurement_af12752f53bfd297b66b504170b9e3655}{get\_initial\_timestamp}():
00167         \textcolor{keywordflow}{return} execution\_time\_measurement.initial\_timestamp
00168     \textcolor{comment}{# ============================================================}
00169     \textcolor{comment}{##  Method to determine the elapsed time from the initial}
00170     \textcolor{comment}{#       timestamp.}
00171     \textcolor{comment}{#   @param type\_timestamp - Indicates if either of the following}
00172     \textcolor{comment}{#               methods of performance measurement is preferred.}
00173     \textcolor{comment}{#               * perf\_counter, perf\_counter(): pc\_timestamp()}
00174     \textcolor{comment}{#               * process\_time, process\_time(): pt\_timestamp()}
00175     \textcolor{comment}{#               * time, time.time\_ns(): time\_ns()}
00176     \textcolor{comment}{#               * monotonic, monotonic(): pm\_monotonic()}
00177     \textcolor{comment}{#   @return the elapsed time from the initial timestamp.}
00178     \textcolor{comment}{#   O(1) method.}
00179     @staticmethod
\hypertarget{performance__measurement__no__ns_8py_source_l00180}{}\hyperlink{classutilities_1_1timing__measurements_1_1performance__measurement__no__ns_1_1execution__time__measurement_a465918aa8dcf663887149cbf9a7306b9}{00180}     \textcolor{keyword}{def }\hyperlink{classutilities_1_1timing__measurements_1_1performance__measurement__no__ns_1_1execution__time__measurement_a465918aa8dcf663887149cbf9a7306b9}{get\_elapsed\_time}(type\_timestamp="monotonic"):
00181         \textcolor{stringliteral}{"""}
00182 \textcolor{stringliteral}{            Is the option for one of the following methods to measure}
00183 \textcolor{stringliteral}{                time?}
00184 \textcolor{stringliteral}{                * perf\_counter, perf\_counter(): pc\_timestamp()}
00185 \textcolor{stringliteral}{                * process\_time, process\_time(): pt\_timestamp()}
00186 \textcolor{stringliteral}{                * time, time.time\_ns(): time\_ns()}
00187 \textcolor{stringliteral}{                * monotonic, monotonic(): pm\_monotonic()}
00188 \textcolor{stringliteral}{        """}
00189         \textcolor{keywordflow}{if} (\textcolor{stringliteral}{"perf\_counter"} == type\_timestamp):
00190             \textcolor{comment}{# Yes. Use perf\_counter() to measure performance/time.}
00191             current\_timestamp = pc\_timestamp()
00192         \textcolor{keywordflow}{elif} (\textcolor{stringliteral}{"process\_time"} == type\_timestamp):
00193             \textcolor{comment}{# Yes. Use process\_time() to measure performance/time.}
00194             current\_timestamp = pt\_timestamp()
00195         \textcolor{keywordflow}{elif} (\textcolor{stringliteral}{"time"} == type\_timestamp):
00196             \textcolor{comment}{# Yes. Use time.time() to measure performance/time.}
00197             current\_timestamp = time.time()
00198         \textcolor{keywordflow}{else}:
00199             \textcolor{stringliteral}{"""}
00200 \textcolor{stringliteral}{                The default option is: "monotonic".}
00201 \textcolor{stringliteral}{                Use monotonic() to measure performance/time.}
00202 \textcolor{stringliteral}{            """}
00203             current\_timestamp = pm\_monotonic()
00204         \textcolor{keywordflow}{return} (current\_timestamp - execution\_time\_measurement.get\_initial\_timestamp())
00205     \textcolor{comment}{# ============================================================}
00206     \textcolor{comment}{##  Method to compare techniques for measuring elapsed periods.}
00207     \textcolor{comment}{#   It calculates the factorial of each number in a list, and}
00208     \textcolor{comment}{#       uses each of the following methods of performance}
00209     \textcolor{comment}{#       measurement to measure the elapsed periods.}
00210     \textcolor{comment}{#       * perf\_counter, perf\_counter(): pc\_timestamp()}
00211     \textcolor{comment}{#       * process\_time, process\_time(): pt\_timestamp()}
00212     \textcolor{comment}{#       * time, time.time(): time()}
00213     \textcolor{comment}{#       * monotonic, monotonic(): pm\_monotonic()}
00214     \textcolor{comment}{#   @return - Nothing.}
00215     \textcolor{comment}{#   O(n!) method, where n is the largest number in the}
00216     \textcolor{comment}{#       aforementioned list, since we are measuring the}
00217     \textcolor{comment}{#       performance of calculating factorials.}
00218     @staticmethod
\hypertarget{performance__measurement__no__ns_8py_source_l00219}{}\hyperlink{classutilities_1_1timing__measurements_1_1performance__measurement__no__ns_1_1execution__time__measurement_a75eea39203f4d9b779977f9203f6d745}{00219}     \textcolor{keyword}{def }\hyperlink{classutilities_1_1timing__measurements_1_1performance__measurement__no__ns_1_1execution__time__measurement_a75eea39203f4d9b779977f9203f6d745}{compare\_different\_methods\_to\_measure\_elapsed\_periods}
      ():
00220         \textcolor{stringliteral}{"""}
00221 \textcolor{stringliteral}{            Create a file to store experimental data of measuring}
00222 \textcolor{stringliteral}{                the performance of recursive and iterative methods.}
00223 \textcolor{stringliteral}{        """}
00224         with open(\textcolor{stringliteral}{"compare\_different\_methods\_to\_measure\_elapsed\_periods.csv"},\textcolor{stringliteral}{"a+"}) \textcolor{keyword}{as} op\_f\_obj:
00225             \textcolor{keywordflow}{for} perf\_measurement\_technique \textcolor{keywordflow}{in} 
      execution\_time\_measurement.types\_of\_performance\_measurement\_technique:
00226                 print(\textcolor{stringliteral}{"The technique used is:"},perf\_measurement\_technique,\textcolor{stringliteral}{"="})
00227                 \textcolor{stringliteral}{"""}
00228 \textcolor{stringliteral}{                    Set the initial timestamp for calculating the}
00229 \textcolor{stringliteral}{                        factorial of numbers via recursion.}
00230 \textcolor{stringliteral}{                """}
00231                 execution\_time\_measurement.set\_initial\_timestamp(perf\_measurement\_technique)
00232                 print(\textcolor{stringliteral}{" Calculate the factorial using recursion."})
00233                 print(\textcolor{stringliteral}{" = current timestamp:"},execution\_time\_measurement.get\_initial\_timestamp(),\textcolor{stringliteral}{"="})
00234                 \textcolor{keywordflow}{for} x \textcolor{keywordflow}{in} range(0,20+1):
00235                     print(\textcolor{stringliteral}{"     factorial of"},x,\textcolor{stringliteral}{" is:"},calculate\_factorial.get\_factorial\_recursion(x),\textcolor{stringliteral}{"="})
00236                 \textcolor{stringliteral}{"""}
00237 \textcolor{stringliteral}{                    Get the elapsed time for calculating the factorial of}
00238 \textcolor{stringliteral}{                        numbers via recursion.}
00239 \textcolor{stringliteral}{                """}
00240                 elapsed\_time\_recursion = execution\_time\_measurement.get\_elapsed\_time(
      perf\_measurement\_technique)
00241                 print(\textcolor{stringliteral}{" = elapsed\_time\_recursion:"},elapsed\_time\_recursion,\textcolor{stringliteral}{"="})
00242                 \textcolor{stringliteral}{"""}
00243 \textcolor{stringliteral}{                    Set the initial timestamp for calculating the}
00244 \textcolor{stringliteral}{                        factorial of numbers via iteration.}
00245 \textcolor{stringliteral}{                """}
00246                 execution\_time\_measurement.set\_initial\_timestamp(perf\_measurement\_technique)
00247                 print(\textcolor{stringliteral}{" Calculate the factorial using iteration."})
00248                 print(\textcolor{stringliteral}{" = current timestamp:"},execution\_time\_measurement.get\_initial\_timestamp(),\textcolor{stringliteral}{"="})
00249                 \textcolor{keywordflow}{for} x \textcolor{keywordflow}{in} range(0,20+1):
00250                     print(\textcolor{stringliteral}{"     factorial of"},x,\textcolor{stringliteral}{" is:"},calculate\_factorial.get\_factorial\_iteration(x),\textcolor{stringliteral}{"="})
00251                 \textcolor{stringliteral}{"""}
00252 \textcolor{stringliteral}{                    Get the elapsed time for calculating the factorial of}
00253 \textcolor{stringliteral}{                        numbers via iteration.}
00254 \textcolor{stringliteral}{                """}
00255                 elapsed\_time\_iteration = execution\_time\_measurement.get\_elapsed\_time(
      perf\_measurement\_technique)
00256                 print(\textcolor{stringliteral}{" = elapsed\_time\_iteration:"},elapsed\_time\_iteration,\textcolor{stringliteral}{"="})
00257                 \textcolor{stringliteral}{"""}
00258 \textcolor{stringliteral}{                    The timeit.timeit() method can result in negative elapsed time.}
00259 \textcolor{stringliteral}{                """}
00260                 text = perf\_measurement\_technique + \textcolor{stringliteral}{","} + str(elapsed\_time\_recursion) + \textcolor{stringliteral}{","} + str(
      elapsed\_time\_iteration) + \textcolor{stringliteral}{"\(\backslash\)n"}
00261                 op\_f\_obj.write(text)
00262                 \textcolor{comment}{#op\_f\_obj.write("\(\backslash\)n")}
00263 
00264 
00265 \textcolor{comment}{###############################################################}
00266 \textcolor{comment}{# Main method for the program.}
00267 
00268 \textcolor{comment}{#   If this is executed as a Python script,}
00269 \textcolor{keywordflow}{if} \_\_name\_\_ == \textcolor{stringliteral}{"\_\_main\_\_"}:
00270     print(\textcolor{stringliteral}{"=================================================="})
00271     print(\textcolor{stringliteral}{"Compare techniques for measuring elapsed periods."})
00272     print(\textcolor{stringliteral}{""})
00273     execution\_time\_measurement.compare\_different\_methods\_to\_measure\_elapsed\_periods()
\end{DoxyCode}
