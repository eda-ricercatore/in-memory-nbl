\hypertarget{get__factorial_8py_source}{}\section{get\+\_\+factorial.\+py}
\label{get__factorial_8py_source}\index{utilities/timing\+\_\+measurements/get\+\_\+factorial.\+py@{utilities/timing\+\_\+measurements/get\+\_\+factorial.\+py}}

\begin{DoxyCode}
\hypertarget{get__factorial_8py_source_l00001}{}\hyperlink{namespaceutilities_1_1timing__measurements_1_1get__factorial}{00001} \textcolor{comment}{#!/Users/zhiyang/anaconda3/bin/python3}
00002 
00003 \textcolor{stringliteral}{"""}
00004 \textcolor{stringliteral}{    This Python script is written by Zhiyang Ong to calculate}
00005 \textcolor{stringliteral}{        the factorial of a number.}
00006 \textcolor{stringliteral}{}
00007 \textcolor{stringliteral}{    The results for this script are compared to the table of}
00008 \textcolor{stringliteral}{        factorials in \(\backslash\)cite\{Pierce2019\}, and validated/verified}
00009 \textcolor{stringliteral}{        for factorials from 0.}
00010 \textcolor{stringliteral}{    This script }
00011 \textcolor{stringliteral}{}
00012 \textcolor{stringliteral}{    Synopsis:}
00013 \textcolor{stringliteral}{    Calculate the factorial of a number.}
00014 \textcolor{stringliteral}{}
00015 \textcolor{stringliteral}{    This script can be executed as follows:}
00016 \textcolor{stringliteral}{    ./get\_factorial.py [a number]}
00017 \textcolor{stringliteral}{}
00018 \textcolor{stringliteral}{    Parameters:}
00019 \textcolor{stringliteral}{    [a number]:     A number that the user wants to determine the}
00020 \textcolor{stringliteral}{                        factorial of.}
00021 \textcolor{stringliteral}{}
00022 \textcolor{stringliteral}{}
00023 \textcolor{stringliteral}{    Revision History:}
00024 \textcolor{stringliteral}{    September 6, 2019           Version 0.1 Script.}
00025 \textcolor{stringliteral}{}
00026 \textcolor{stringliteral}{    References:}
00027 \textcolor{stringliteral}{        [Pierce2019]}
00028 \textcolor{stringliteral}{            Rod Pierce, "Factorial Function," from Maths Is Fun, 2019. Available online from "Maths Is Fun:
       Numbers" at: https://www.mathsisfun.com/numbers/factorial.html; September 19, 2019 is the last access date.}
00029 \textcolor{stringliteral}{                [No address]}
00030 \textcolor{stringliteral}{                https://www.mathsisfun.com/citation.php}
00031 \textcolor{stringliteral}{"""}
00032 
\hypertarget{get__factorial_8py_source_l00033}{}\hyperlink{namespaceutilities_1_1timing__measurements_1_1get__factorial_a043c91cb8f16c3570870212cabbde3c1}{00033} \_\_author\_\_ = \textcolor{stringliteral}{'Zhiyang Ong'}
\hypertarget{get__factorial_8py_source_l00034}{}\hyperlink{namespaceutilities_1_1timing__measurements_1_1get__factorial_af4911964913b4a07e464413868da36f7}{00034} \_\_version\_\_ = \textcolor{stringliteral}{'1.0'}
\hypertarget{get__factorial_8py_source_l00035}{}\hyperlink{namespaceutilities_1_1timing__measurements_1_1get__factorial_a8b8b3f7bee60cdc7d63a3f6357047a30}{00035} \_\_date\_\_ = \textcolor{stringliteral}{'September 6, 2019'}
00036 
00037 \textcolor{comment}{#   The MIT License (MIT)}
00038 
00039 \textcolor{comment}{#   Copyright (c) <2019> <Zhiyang Ong>}
00040 
00041 \textcolor{comment}{#   Permission is hereby granted, free of charge, to any person obtaining a copy of this software and
       associated documentation files (the "Software"), to deal in the Software without restriction, including without
       limitation the rights to use, copy, modify, merge, publish, distribute, sublicense, and/or sell copies of the
       Software, and to permit persons to whom the Software is furnished to do so, subject to the following
       conditions:}
00042 
00043 \textcolor{comment}{#   The above copyright notice and this permission notice shall be included in all copies or substantial
       portions of the Software.}
00044 
00045 \textcolor{comment}{#   THE SOFTWARE IS PROVIDED "AS IS", WITHOUT WARRANTY OF ANY KIND, EXPRESS OR IMPLIED, INCLUDING BUT NOT
       LIMITED TO THE WARRANTIES OF MERCHANTABILITY, FITNESS FOR A PARTICULAR PURPOSE AND NONINFRINGEMENT. IN NO
       EVENT SHALL THE AUTHORS OR COPYRIGHT HOLDERS BE LIABLE FOR ANY CLAIM, DAMAGES OR OTHER LIABILITY, WHETHER IN AN
       ACTION OF CONTRACT, TORT OR OTHERWISE, ARISING FROM, OUT OF OR IN CONNECTION WITH THE SOFTWARE OR THE USE
       OR OTHER DEALINGS IN THE SOFTWARE.}
00046 
00047 \textcolor{comment}{#   Email address: echo "cukj -wb- 23wU4X5M589 TROJANS cqkH wiuz2y 0f Mw Stanford" | awk '\{
       sub("23wU4X5M589","F.d\_c\_b. ") sub("Stanford","d0mA1n"); print $5, $2, $8; for (i=1; i<=1; i++) print "6\(\backslash\)b"; print $9, $7,
       $6 \}' | sed y/kqcbuHwM62z/gnotrzadqmC/ | tr 'q' ' ' | tr -d [:cntrl:] | tr -d 'ir' | tr y "\(\backslash\)n"   Che cosa
       significa?}
00048 
00049 
00050 \textcolor{comment}{###############################################################}
00051 \textcolor{stringliteral}{"""}
00052 \textcolor{stringliteral}{    Import modules from The Python Standard Library.}
00053 \textcolor{stringliteral}{    sys         Get access to any command-line arguments.}
00054 \textcolor{stringliteral}{    os          Use any operating system dependent functionality.}
00055 \textcolor{stringliteral}{    os.path     For pathname manipulations.}
00056 \textcolor{stringliteral}{}
00057 \textcolor{stringliteral}{    subprocess -> call}
00058 \textcolor{stringliteral}{                To make system calls.}
00059 \textcolor{stringliteral}{    time        To measure elapsed time.}
00060 \textcolor{stringliteral}{    warnings    Raise warnings.}
00061 \textcolor{stringliteral}{    re          Use regular expressions.}
00062 \textcolor{stringliteral}{}
00063 \textcolor{stringliteral}{    pathlib->Path}
00064 \textcolor{stringliteral}{                For mapping a string to a path.}
00065 \textcolor{stringliteral}{    datetime    To obtain information about the current date and time.}
00066 \textcolor{stringliteral}{    time    To obtain information about the current time.}
00067 \textcolor{stringliteral}{    warnings    Provide warnings to users without terminating the}
00068 \textcolor{stringliteral}{                    program abruptly.}
00069 \textcolor{stringliteral}{"""}
00070 
00071 \textcolor{keyword}{import} sys
00072 \textcolor{keyword}{import} os
00073 \textcolor{keyword}{import} os.path
00074 \textcolor{comment}{#from pathlib import Path}
00075 \textcolor{keyword}{from} subprocess \textcolor{keyword}{import} call
00076 \textcolor{keyword}{import} time
00077 \textcolor{keyword}{import} warnings
00078 \textcolor{keyword}{import} re
00079 \textcolor{keyword}{import} datetime
00080 \textcolor{keyword}{import} time
00081 \textcolor{keyword}{import} warnings
00082 
00083 \textcolor{comment}{###############################################################}
00084 \textcolor{stringliteral}{"""}
00085 \textcolor{stringliteral}{    Module with methods that determine the factorial of a number.}
00086 \textcolor{stringliteral}{"""}
\hypertarget{get__factorial_8py_source_l00087}{}\hyperlink{classutilities_1_1timing__measurements_1_1get__factorial_1_1calculate__factorial}{00087} \textcolor{keyword}{class }\hyperlink{classutilities_1_1timing__measurements_1_1get__factorial_1_1calculate__factorial}{calculate\_factorial}:
00088     \textcolor{comment}{# Number to determine the factorial of.}
\hypertarget{get__factorial_8py_source_l00089}{}\hyperlink{classutilities_1_1timing__measurements_1_1get__factorial_1_1calculate__factorial_a52e3407bbd93719d9c77e49706a47362}{00089}     default\_number = 10
00090     \textcolor{comment}{# Number to compute the factorial of.}
\hypertarget{get__factorial_8py_source_l00091}{}\hyperlink{classutilities_1_1timing__measurements_1_1get__factorial_1_1calculate__factorial_a04cd527a2af28e713e8665e2d40b1fae}{00091}     number\_to\_compute = 10
00092     \textcolor{stringliteral}{"""}
00093 \textcolor{stringliteral}{        Number to indicate that the factorial for the given input}
00094 \textcolor{stringliteral}{            does not exist.}
00095 \textcolor{stringliteral}{    """}
\hypertarget{get__factorial_8py_source_l00096}{}\hyperlink{classutilities_1_1timing__measurements_1_1get__factorial_1_1calculate__factorial_a16363bc4d672d6352ce3618abeb8e74f}{00096}     does\_not\_exist = -1234567890
00097     \textcolor{comment}{# ============================================================}
00098     \textcolor{comment}{##  Method to process the optional input argument.}
00099     \textcolor{comment}{#   If the number is not provided, use the value of "default\_number".}
00100     \textcolor{comment}{#   @return - Nothing.}
00101     \textcolor{comment}{#   O(1) method.}
00102     @staticmethod
\hypertarget{get__factorial_8py_source_l00103}{}\hyperlink{classutilities_1_1timing__measurements_1_1get__factorial_1_1calculate__factorial_a799aaf842a0c6c98d31ee1cad8381caa}{00103}     \textcolor{keyword}{def }\hyperlink{classutilities_1_1timing__measurements_1_1get__factorial_1_1calculate__factorial_a799aaf842a0c6c98d31ee1cad8381caa}{process\_optional\_input\_argument}():
00104         \textcolor{comment}{# If the path to the README file is not specified}
00105         \textcolor{keywordflow}{if} (2 > len(sys.argv)):
00106             \textcolor{stringliteral}{"""}
00107 \textcolor{stringliteral}{                The optional input argument is not provided by the}
00108 \textcolor{stringliteral}{                    user.}
00109 \textcolor{stringliteral}{                We assume that the default number to determine the}
00110 \textcolor{stringliteral}{                    factorial of is indicated by the value of}
00111 \textcolor{stringliteral}{                    "default\_number".}
00112 \textcolor{stringliteral}{            """}
00113             calculate\_factorial.number\_to\_compute = calculate\_factorial.default\_number
00114         \textcolor{keywordflow}{else}:
00115             calculate\_factorial.number\_to\_compute = sys.argv[1]
00116     \textcolor{comment}{# ============================================================}
00117     \textcolor{comment}{##  Method to determine the factorial of "number\_to\_compute"}
00118     \textcolor{comment}{#       by recursion.}
00119     \textcolor{comment}{#   @param given\_number - Number to determine the factorial of.}
00120     \textcolor{comment}{#   @return the factorial of given\_number (if it is a non-negative}
00121     \textcolor{comment}{#       integer);}
00122     \textcolor{comment}{#       else, return 'None'.}
00123     \textcolor{comment}{#   O(n) method, where n is the number of test cases used.}
00124     @staticmethod
\hypertarget{get__factorial_8py_source_l00125}{}\hyperlink{classutilities_1_1timing__measurements_1_1get__factorial_1_1calculate__factorial_ad985135073d51b522bdfb4c5b5569456}{00125}     \textcolor{keyword}{def }\hyperlink{classutilities_1_1timing__measurements_1_1get__factorial_1_1calculate__factorial_ad985135073d51b522bdfb4c5b5569456}{get\_factorial\_recursion}(given\_number):
00126         \textcolor{keywordflow}{if} isinstance(given\_number, int):
00127             \textcolor{keywordflow}{if} (0 == (given\_number) \textcolor{keywordflow}{or} (1 == given\_number)):
00128                 \textcolor{keywordflow}{return} 1
00129             \textcolor{keywordflow}{elif} (0 > given\_number):
00130                 warnings.warn(\textcolor{stringliteral}{"The factorial of a negative number cannot be determined."})
00131                 \textcolor{keywordflow}{return} \textcolor{keywordtype}{None}
00132             \textcolor{keywordflow}{else}:
00133                 \textcolor{keywordflow}{return} given\_number * calculate\_factorial.get\_factorial\_recursion(given\_number - 1)
00134         \textcolor{keywordflow}{elif} isinstance(given\_number, float):
00135             warnings.warn(\textcolor{stringliteral}{"The factorial of a floating-point number cannot be determined."})
00136             \textcolor{keywordflow}{return} \textcolor{keywordtype}{None}
00137         \textcolor{keywordflow}{else}:
00138             warnings.warn(\textcolor{stringliteral}{"The factorial of a non-integer cannot be determined."})
00139             \textcolor{keywordflow}{return} \textcolor{keywordtype}{None}
00140     \textcolor{comment}{# ============================================================}
00141     \textcolor{comment}{##  Method to determine the factorial of "number\_to\_compute"}
00142     \textcolor{comment}{#       by iteration.}
00143     \textcolor{comment}{#   @param given\_number - Number to determine the factorial of.}
00144     \textcolor{comment}{#   @return the factorial of given\_number (if it is a non-negative}
00145     \textcolor{comment}{#       integer);}
00146     \textcolor{comment}{#       else, return 'None'.}
00147     \textcolor{comment}{#   O(n) method, where n is the number of test cases used.}
00148     @staticmethod
\hypertarget{get__factorial_8py_source_l00149}{}\hyperlink{classutilities_1_1timing__measurements_1_1get__factorial_1_1calculate__factorial_a4233e6dad246c88e546c31f5752a1ee1}{00149}     \textcolor{keyword}{def }\hyperlink{classutilities_1_1timing__measurements_1_1get__factorial_1_1calculate__factorial_a4233e6dad246c88e546c31f5752a1ee1}{get\_factorial\_iteration}(given\_number):
00150         \textcolor{keywordflow}{if} isinstance(given\_number, int):
00151             \textcolor{keywordflow}{if} (0 == (given\_number) \textcolor{keywordflow}{or} (1 == given\_number)):
00152                 \textcolor{keywordflow}{return} 1
00153             \textcolor{keywordflow}{elif} (0 > given\_number):
00154                 warnings.warn(\textcolor{stringliteral}{"The factorial of a negative number cannot be determined."})
00155                 \textcolor{keywordflow}{return} \textcolor{keywordtype}{None}
00156             \textcolor{keywordflow}{else}:
00157                 result = 1
00158                 \textcolor{keywordflow}{while} (1 < given\_number):
00159                     result = result * given\_number
00160                     given\_number = given\_number - 1
00161                 \textcolor{keywordflow}{return} result
00162         \textcolor{keywordflow}{elif} isinstance(given\_number, float):
00163             warnings.warn(\textcolor{stringliteral}{"The factorial of a floating-point number cannot be determined."})
00164             \textcolor{keywordflow}{return} \textcolor{keywordtype}{None}
00165         \textcolor{keywordflow}{else}:
00166             warnings.warn(\textcolor{stringliteral}{"The factorial of a non-integer cannot be determined."})
00167             \textcolor{keywordflow}{return} \textcolor{keywordtype}{None}
00168 
00169 \textcolor{comment}{###############################################################}
00170 \textcolor{comment}{# Main method for the program.}
00171 
00172 \textcolor{comment}{#   If this is executed as a Python script,}
00173 \textcolor{keywordflow}{if} \_\_name\_\_ == \textcolor{stringliteral}{"\_\_main\_\_"}:
00174     print(\textcolor{stringliteral}{"=================================================="})
00175     print(\textcolor{stringliteral}{"Calculate the factorial of a given number."})
00176     \textcolor{comment}{# get\_factorial\_iteration() requires and only accepts 1 input value.}
00177     \textcolor{comment}{#calculate\_factorial.process\_optional\_input\_argument()}
00178     \textcolor{comment}{#calculate\_factorial.process\_optional\_input\_argument(324324 23r23 4e5678 56789)}
00179     \textcolor{comment}{# Change process\_optional\_input\_argument() to accept no input.}
00180     calculate\_factorial.process\_optional\_input\_argument()
00181     \textcolor{comment}{#print("get\_factorial\_iteration() for default value of
       10:",calculate\_factorial.get\_factorial\_iteration(4),"=")}
00182     print(\textcolor{stringliteral}{"=    Test the factorial computation method using iteration."})
00183     print(\textcolor{stringliteral}{"get\_factorial\_iteration(4):"},calculate\_factorial.get\_factorial\_iteration(4),\textcolor{stringliteral}{"="})
00184     \textcolor{comment}{# ValueError: invalid literal for int() with base 10: 'my string'}
00185     print(\textcolor{stringliteral}{"get\_factorial\_iteration('my string'):"},calculate\_factorial.get\_factorial\_iteration(\textcolor{stringliteral}{"my string"}),\textcolor{stringliteral}{
      "="})
00186     print(\textcolor{stringliteral}{"get\_factorial\_iteration(125.23429):"},calculate\_factorial.get\_factorial\_iteration(125.23429),\textcolor{stringliteral}{"="})
00187     print(\textcolor{stringliteral}{"get\_factorial\_iteration(None):"},calculate\_factorial.get\_factorial\_iteration(\textcolor{keywordtype}{None}),\textcolor{stringliteral}{"="})
00188     print(\textcolor{stringliteral}{"get\_factorial\_iteration(-345):"},calculate\_factorial.get\_factorial\_iteration(-345),\textcolor{stringliteral}{"="})
00189     \textcolor{stringliteral}{"""}
00190 \textcolor{stringliteral}{        Add whitespace before testing the factorial computation}
00191 \textcolor{stringliteral}{            method using recursion.}
00192 \textcolor{stringliteral}{    """}
00193     print(\textcolor{stringliteral}{"\(\backslash\)n\(\backslash\)n"})
00194     \textcolor{comment}{#print("")}
00195     print(\textcolor{stringliteral}{"=    Test the factorial computation method using recursion."})
00196     print(\textcolor{stringliteral}{"get\_factorial\_recursion(4):"},calculate\_factorial.get\_factorial\_recursion(4),\textcolor{stringliteral}{"="})
00197     \textcolor{comment}{# ValueError: invalid literal for int() with base 10: 'my string'}
00198     print(\textcolor{stringliteral}{"get\_factorial\_recursion('my string'):"},calculate\_factorial.get\_factorial\_recursion(\textcolor{stringliteral}{"my string"}),\textcolor{stringliteral}{
      "="})
00199     print(\textcolor{stringliteral}{"get\_factorial\_recursion(125.23429):"},calculate\_factorial.get\_factorial\_recursion(125.23429),\textcolor{stringliteral}{"="})
00200     print(\textcolor{stringliteral}{"get\_factorial\_recursion(None):"},calculate\_factorial.get\_factorial\_recursion(\textcolor{keywordtype}{None}),\textcolor{stringliteral}{"="})
00201     print(\textcolor{stringliteral}{"get\_factorial\_recursion(-345):"},calculate\_factorial.get\_factorial\_recursion(-345),\textcolor{stringliteral}{"="})
00202     
\end{DoxyCode}
